\documentclass[12pt]{article}

\usepackage{amsmath, graphicx, filecontents, url}
\begin{filecontents}{shortbib.bib}
@book{Boyer,
  author={C. B. Boyer and U. C. Merzbach},
  title={A History of Mathematics},
  publisher={Wiley},
  year={2011},
  month={January},
  edition={3}
}
@book{Chemla,
  author={K. Chemla},
  title={The History of Mathematical Proof in Ancient Traditions},
  publisher={Cambridge University Press},
  year={2012},
  month={August},
  edition={1}
}
@misc{O'Connor,
  author={J. J. O'Connor and E. F. Robertson},
  title={Nine Chapters on the Mathematical Art},
  year={2003},
  month={December},
  howpublished={\url{http://www-groups.dcs.st-and.ac.uk/history/HistTopics/Nine\_chapters.html}}
}
@article{Mueller,
  jstor_articletype = {research-article},
  title = {Euclid's Elements and the Axiomatic Method},
  author = {I. Mueller},
  journal = {The British Journal for the Philosophy of Science},
  jstor_issuetitle = {},
  volume = {20},
  number = {4},
  jstor_formatteddate = {Dec., 1969},
  pages = {pp. 289-309},
  url = {http://www.jstor.org/stable/686258},
  ISSN = {00070882},
  abstract = {},
  language = {English},
  year = {1969},
  publisher = {Oxford University Press on behalf of The British Society for the Philosophy of Science},
  copyright = {Copyright © 1969 The British Society for the Philosophy of Science},
}
@misc{Joyce,
  author={D. E. Joyce},
  title={Euclid's Elements},
  year={1998},
  howpublished={\url{http://aleph0.clarku.edu/~djoyce/java/elements/Euclid.html}}
}
@book{Shen,
  title = {The Nine Chapters on the Mathematical Art: Companion and Commentary},
  isbn = {0198539363  9780198539360  7030061012  9787030061010},
  shorttitle = {The Nine Chapters on the Mathematical Art},
  language = {Translated from the Chinese.},
  publisher = {Oxford University Press; Science Press},
  author = {K. Shen, J. N. Crossley, and A. W.-C Lun}, 
  year = {1999}
}
@misc{Wagner,
  author={W. B. Wagner},
  title={A proof of the Pythagorean Theorem by {Liu Hui}
  (third century {AD})},
  howpublished={\url{http://donwagner.dk/Pythagoras/Pythagoras.html}}
}





\end{filecontents}

\title{The Origins of Algorithms: West v. East}
\date{}
\author{}


\begin{document}
\maketitle{}
The algorithm appears to be one of the most primordial concepts available
to mathematicians. It is this fundamental concepts that allow for the
repeated abstractions that allow us to continually progress in mathematics
without having to rediscover the knowledge of our predescessors. It is easy to
forget that our notion of algorithms is in fact the result of thousands of years 
of development, driven by culture, philosophy and history. In fact, it once took on
radically different forms in the ancient western and eastern worlds.

Two of the greatest influencers of our modern school of thought, are the 
oriental mathematics of Ancient Asia, and the western mathematics harking
from the Greeks. Interestingly, they both had canonical texts on Algorithms,
these being \emph{The Nine Chapters on the Mathematical Art} and \emph{Euclid's Elements} 
respectively. Despite the fact that both these texts discovered techniques to 
solve many of the same problems, they featured fundamentally different philosophies towards mathematics.

It is ideal therefore, to conduct a comparative study of the two texts. After analysis of their
respective algorithms and proofs, it can be concluded that early western mathematics
had a proof-first, epistemic philosophy, whereas early eastern mathematics was defined by an algorithm-first,
pragmatic approach.

\section{Historical Background}
\subsection{The Nine Chapters}
The \emph{Nine Chapters on the Mathematical Art} 
 (hereby referred to in this papers as \emph{Nine Chapters})
is well regarded as the canonical text of Ancient
Chinese Mathematics. References to it throughout
history indicate that it was the dominant
authoritative text on mathematics not only in China
but throughout Asia until around the 14th Century
CE. It served not only as the leading textbook for
education in mathematics, but also as the
cornerstone work upon which other mathematical
work was built. In this sense, it served a role
similar to that of \emph{Euclid’s Elements}~\cite{Boyer}.

The text is difficult to date as it was
originally produced on bamboo strips, none of
which have survived. Most historians estimate that
the problems and methods in the text were
completed around 100B.C., with further
commentary added later~\cite{Chemla}.
In total, the book includes 246 problems divided
into chapters based on subject matter. Most of the
text focuses on practical problems of the day,
including surveying, goods exchange, engineering,
and right-angled triangles~\cite{O'Connor}.
Each problem is presented as an imagined
scenario; it describes the situation with real world
terms (doors, fields, grain) and then asks the reader
to `Tell,’ requesting some specific answer to a
problem posed.

In the tradition of Chinese classical texts, the
Nine Chapters has no specific author, but rather
follows the ancient collaborative tradition in which
many writers build upon the work of their
predecessors~\cite{Boyer}. However, Liu Hui, a
Chinese mathematician born around 220 AD, is of
special notability as his commentary in the Nine
Chapters provided many of the proofs that we are
to analyze in this paper.

\subsection{Euclid's Elements}

\emph{Euclid’s Elements} (referred to in this paper as
\emph{Elements}) is often regarded amongst the foundation of
modern mathematics. For more than 2000 years
after its creation, it was regarded as the cornerstone
text on the fundamental principles of mathematics,
including an extensive discourse on elementary
arithmetic, geometry and algebra~\cite{Boyer}. The
Greek philosopher Proclus likened the relationship
between \emph{Elements} and the rest of Mathematics to
that between the letters of the alphabet and
language itself. One of its greatest innovations and
the source of much of its praise was Euclid’s use of
the `axiomatic method,' which has now established
itself at the core of modern mathematics~\cite{Mueller}.

Euclid of Alexandria, a Greek mathematician,
originally authored \emph{Elements} around 300 BC.
Though Euclid is often the sole author to whom the
\emph{Elements} are attributed, he himself made no claim
to originality. It is believed that Euclid drew heavily
upon the work of his predecessors, though the
arrangement of \emph{Elements} and many of its proofs
were supplied by Euclid himself.

\emph{Elements} is partitioned into thirteen chapters,
which cover plane geometry, number theory, solid
geometry and incommensurables. The axiomatic
style is clear, as \emph{Elements} has no known preface
and rather begins with twenty-three definitions
followed by five axioms and five postulates. It is
from this basis that Euclid begins the rigorous
mathematical discourse that covers over 450
propositions, each one progressing in small steps,
building upon knowledge established in previous
episodes~\cite{Joyce}.

\section{Analysis of Algorithms and Proofs}

The three problems to be comparatively
analyzed in both \emph{Elements} and \emph{Nine Chapters} are
(a) Pythagoras’ theorem, (b) square roots and (c)
quadratic equations. These three problems were
chosen because of the fundamental role they play
in modern and ancient mathematics. Furthermore,
they are all well-defined problems, which allow us
to confidently identify equivalent sections of both
texts, allowing for a more specific comparative
study. Finally, the problems are addressed in
different sections throughout both texts, and
together they represent a good cross sectional
representation of the style and form of both
\emph{Elements} and \emph{Nine Chapters}.

\subsection{Pythagoras' Theorem}
A starting point for our investigation is an analysis
of Pythagoras’ theorem; that the square of the
length of the hypotenuse of the triangle is equal to
the sum of the square of the other two sides.
Pythagoras’ theorem has played a
prominent role in most ancient mathematical
systems.
the theorem covers problems in the form:

\begin{equation}
  a^2 + b^2 = c^2
\end{equation}

\subsubsection{\emph{Nine Chapters:} Chapter Nine, Problems 1--3}
The Pythagoras rule was well known in China
under the name of the ‘gougu rule.’ The rule
states:
\begin{quote}
  Add the squares of the gou and the gu, take the
  square root of the sum giving the hypotenuse’.
  Further, if the gu is subtracted from the square on
  the hypotenuse. The square root of the remainder
  is the gou. Further, if the gou is subtracted from
  the square of the hypotenuse. The square root of
  the remainder is the gu.
\end{quote}
Liu Hui's commentary adds that:
\begin{quote}
  The gou is shorter than the gu. The gou is
  shorter than the hypotenuse. They apply in
  various problems in terms of rates of proportion
\end{quote}

Here already we can see a strong understanding
of algebra, in that one rule is sufficient to draw
three corollaries from. The explanation of the
gougu rule gives three alternative methods for its
use, and then proves each method to be valid. The
proof is as follows:
\begin{quote}
  Let the square on the gou be red in colour, the
  square on the gu be blue. Let the deficit and
  excess parts be mutually substituted into
  corresponding positions, the other parts remain
  unchanged~\cite{Shen}.
\end{quote}

Liu’s original diagram has since been lost, and
there is ambiguity as its original structure.
However, three academics have attempted valid
reconstructions. Li Huang’s diagram is shown in
Figure 2,~\cite{Shen} whereas Figure 3 is conceived
by Donald B. Wanger, and Figure 4 is suggested by
Professor Jöran Friberg~\cite{Wagner}.












\bibliographystyle{unsrt}
\bibliography{shortbib}

\end{document}

